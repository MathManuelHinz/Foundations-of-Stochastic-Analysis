

\section*{Introduction}

    \begin{itemize}
        \item Information on ecampus
        \item Exercise groups \begin{itemize}
            \item 3 Slots
            \item Register to your preferred slot
            \item Add in the comments: \begin{itemize}
                \item Name(s) of exercise partner students
                \item Groups of 2 or 3 students
                \item Preference of other slots and / or which slots are not possible (why)
            \end{itemize} 
        \end{itemize}
        \item Exercise sheets on ecampus, by Friday
        \item Oral exam \textbf{first} oder third week after lecture period
        \item Most content done by christmas! -> learn for the exam in the break
    \end{itemize}

\section*{Revision: Stochastic processes}

    \begin{itemize}
        \item First part: mostly repetition in continuous time. 
        \item Important: conditional expectation, short review on friday
    \end{itemize}


\section{Introduction}

\subsection{Motivation}

\underline{\textbf{Brownian motion:}} Limit of random walk:

Let $X_1,X_2,\dots$ iid. random variable with $\mathbb{P}(X_i=1)=\mathbb{P}(X_i=-1)=\frac{1}{2}$ and 
\[S_{t,n}=\frac{1}{\sqrt{n}}\sum_{k=1}^{\lfloor tn \rfloor} X_k\]

% Picture random walk

By CLT $S_{t,n}\stackrel{\to\infty}{\to}$ Gaussian random variable.
Brownian motion: $B_t=\lim_{n\to\infty} S_{t,n}$

% N_t in anfürhungszeichen

\begin*{example}[Population growth]
    Let $S_t$ be the size of a population at time $t$, let $R$ be the average growth rate.

    \[\implies \frac{dS_t}{dt}=RS_t\implies S_t=S_0e^{Rt}\]

    There are fluctuations
    \[\implies \frac{dS_t}{dt}=(R+\underbrace{N_t}_{\text{noise term}})S_t\]
\end*{example}

\begin*{example}[Langvin equation]
    Describes the evolution of a small particle in a fluid:
    \[m\frac{dV_t}{dt}=-\eta V_t + \underbrace{N_t}_{\text{noise term}}\]
    Where $m$ is the mass of the particle, $V_t$ is the speed at time t and $\eta$ is the viscosity coefficient.
\end*{example}

\begin*{example}[Dirichlet problems]
    Let $f$ be a harmonic function on a domain $D\subset \R^d$, i.e. $\Delta f = \sum_{k=1}^d \frac{\partial^2}{\partial x_k^2}f = 0$.
    
    If $f$ is kown on $\partial D$:
    \[\implies x\in D\setminus \partial D: f(x)=\mathbb{E}(f(B_{\tau}^x))\]

    where $B_t^x$ is Brownian motion starting at x, $\tau=\inf \{t>0|B_t^x\in \partial D\}$.

\end*{example}

\underline{\textbf{What should $N_t$ be ?}}

\underline{\textbf{1. Trial:}} Take $N_t$ to be a fully random function in continuous
time, i.e., analog of a sequence of random variables.

$\implies $ Ask for 
\begin{enumerate}
    \item $N_t$ is independent of $N_s$ for $t\neq t$
    \item $N_t$ has a distribution independent of $t$
    \item $\mathbb{E}(N_t)=0$
\end{enumerate}

\underline{\textbf{Problem:}}

$N_t$ is not measurable, expect when $N_t=0$ (why?).

Let $\nu$ be the distribution of $N_t$. 

$\implies \exists a\in \R: p=\mathbb{P}(N_t\leq a)\in (0,1)$.
Let $E=\{t\geq 0| N_t\leq a\}$. Therefore $E$ is not Lebesgue-measurable:
If $E$ where measurable  $\implies \forall c<d: \text{leb}(E\cap (c,d))=p\cdot(d-c)$ 
But if measurable: $\forall \alpha < 1\exists (c,d)$ s.t. $\text{leb}(E\cap (c,d))>\alpha \cdot (d-c) \implies$ contradiction.

In the examples, we were interested not directly to $N_t$, but to integrals: 
\[\frac{dS_t}{dt}=(R+N_t)S_t \to \int_0^u \frac{dS_t}{dt}=S_u-S_0=\int_0^u R S_t dt + \int_0^u N_tS_t dt\]

\underline{\textbf{2. Trial:}} Let $B_t\coloneqq \int_0^t N_s ds$.

$\implies $ Ask for :

\begin{enumerate}
    \item[(BM1)] For $0=t_0<t_1<,\dots,t_n$, the random variables \[B_{t_{j+1}}-B_{t_j} j=0,\dots,n-1\] are \textbf{independent}
    \item[(BM2)] $B_t$ has stationary increments: \[B_{t_1+s}-B_{u_1+s},\dots,B_{t_n+s}-B_{u_n+s}\] is independent of $s\geq 0$ for all $u_i<t_i$ for $i=1,\dots,n$ % if they don't overlap
    \item[(BM3)] $\mathbb{E}(B_t)=0$.
\end{enumerate}

Add: \textbf{normalization}:

\begin{enumerate}
    \item[(BM4)] $\mathbb{E}(B_1^2)=1$
\end{enumerate}

Add: \textbf{continuity}

\begin{enumerate}
    \item[(BM5)] $t\mapsto B_t$ is (almost surely) continuous.
\end{enumerate}

We will see that a process with (BM1) - (BM5) exists (Wiener process / Brownian Motion).

\begin{lemma}

    Property (BM5) implies $\forall \epsilon>0$:
    \[\lim_{n\to\infty} \mathbb{P}\left(\left\vert B_{t+\frac{1}{n}}-B_t \right\vert>\epsilon \right)=0\]
\end{lemma}

\begin{proof}
    Implicit assumption: $B_0=0$ (otherwise we have to carry the $B_0$ term)
    Let $H_t\coloneqq \sup_{1\leq k \leq n}\left\vert B_{\frac{k}{n}}-\frac{k-1}{n} \right\vert$
    \[\implies \forall \epsilon >0: \mathbb{P}(H_n)>\epsilon\stackrel{\to \infty}{\rightarrow} 0\]

    But $\mathbb{P}(H_n> \epsilon)=1-\mathbb{P}(H_n \leq \epsilon)=1-\mathbb{P}(\left\vert B_{\frac{1}{n}-B_0} \right\vert,\dots,\left\vert B_{\frac{n}{n}-B_{\frac{n-1}{n}}} \right\vert\leq  \epsilon) $

    \[\stackrel{\text{BM1}}{=}1- \prod_{k=0}^n\mathbb{P}(\left\vert  B_{\frac{k}{n}}-B_{\frac{k-1}{n}} \right\vert \leq \epsilon) \stackrel{\text{BM2}}{=} 1-\mathbb{P}(\left\vert  B_{\frac{1}{n}}-B_{\frac{0}{n}} \right\vert \leq \epsilon)^n \]
    \[=1-(\underbrace{1-\mathbb{P}(\left\vert B_{\frac{1}{n}} \right\vert)>0}_{\leq \exp(-n \mathbb{P}(\left\vert B_{\frac{1}{n}} \right\vert>\epsilon))}) \stackrel{1-x\leq e^{-x}}{=}1-\exp(-n\mathbb{P}(\left\vert B_{\frac{1}{n}} \right\vert>\epsilon))\]

    $\implies n\to \infty, n\mathbb{P}(\left\vert B_{\frac{1}{n}-B_0} \right\vert>\epsilon) \stackrel{n\to\infty}{\rightarrow}0$

\end{proof}

\begin{lemma}
    $\forall s<t:B_t-B_s\sim \mathcal{N}(0,t-s)$, i.e.:
    \[\mathbb{P(B_t-B_s)\leq x}=\frac{1}{\sqrt[2]{2\pi/(t-s)}}\int_{-\infty}^\infty dy e^{-\frac{y}{2/(t-s)}}\]
\end{lemma}

\begin{proof}
    Take $s=0$, w.l.o.g..

    Let $B_t=\sum_{k=0}^n X_{n,k}, X_{n,k}=B_{\frac{kt}{n}-B_{\frac{(k-1)t}{n}}}$ are iid.
    By (BM3) $\implies \mathbb{E}(X_{n,k})=0$ and $\mathbb{E}(B_{\frac{t}{n}})=0$.

    Assume: $\mathbb{E}(B_1^2)=1 \implies \text{Var}(B_1^2)\underbrace{=}_{t=1}=\sum_{k=1}^n\text{Var}(X_{n,k})\implies \text{Var}(X_{n,k})=\frac{t}{n}$
    $\implies \text{Var}(B_t)=t$.

    CLT finishes the prove.

\end{proof}

Let ($\tilde{\text{BM2}}$): For $s,t\geq 0:$ \[\mathbb{P(B_{t+s}-B_s\in A)}=\frac{1}{\sqrt{2 \pi t}}\int_A dxe^{-\frac{x^2}{2t}}\]

\begin{definition}
    A one-dimensional Brownian Motion is a real valued process $B_t,t\geq 0$, s.t.:
    
    (BM1), ($\tilde{\text{BM2}}$),(BM5).
\end{definition}

\begin{remark}
    The standard brownian motion also has $B_0=0$.
\end{remark}

Next: Construct a process satisfying this definition and study its properties. 


